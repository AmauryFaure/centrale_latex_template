% Template:     Informe LaTeX
% Documento:    Configuraciones finales
% Versión:      DEV
% Codificación: UTF-8
%
% Autor: Pablo Pizarro R.
%        Facultad de Ciencias Físicas y Matemáticas
%        Universidad de Chile
%        pablo@ppizarror.com
%
% Manual template: [https://latex.ppizarror.com/informe]
% Licencia MIT:    [https://opensource.org/licenses/MIT]

\newcommand{\templateFinalcfg}{
	
	% -----------------------------------------------------------------------------
	% Se reestablecen headers y footers
	% -----------------------------------------------------------------------------
	\markboth{}{}
	\newpage

	% Actualiza headers
	\ifthenelse{\equal{\disablehfrightmark}{false}}{
		
		% Define funciones generales
		\def\COREhfstyledefA { % 1, 2, 4, 9, 11, 14, 15
			\fancypagestyle{plain}{\fancyhead[L]{\nouppercase{\leftmark}}}
			\fancyhead[L]{\nouppercase{\leftmark}}
		}
		\def\COREhfstyledefB { % 5
			\fancypagestyle{plain}{
				\ifthenelse{\equal{\hfwidthwrap}{true}}{
					\fancyhead[R]{
						\begin{minipage}[t]{\hfwidthtitle\linewidth}
							\begin{flushright}
								\nouppercase{\leftmark}
							\end{flushright}
						\end{minipage}
					}
				}{
					\fancyhead[R]{\nouppercase{\leftmark}}
				}
			}
			\ifthenelse{\equal{\hfwidthwrap}{true}}{
				\fancyhead[R]{
					\begin{minipage}[t]{\hfwidthtitle\linewidth}
						\begin{flushright}
							\nouppercase{\leftmark}
						\end{flushright}
					\end{minipage}
				}
			}{
				\fancyhead[R]{\nouppercase{\leftmark}}
			}
		}
		\def\COREhfstyledefC { % 10
			\fancypagestyle{plain}{
				\ifthenelse{\equal{\hfwidthwrap}{true}}{
					\fancyhead[L]{
						\begin{minipage}[t]{\hfwidthtitle\linewidth}
							\begin{flushleft}
								\nouppercase{\leftmark}
							\end{flushleft}
						\end{minipage}
					}
				}{
					\fancyhead[L]{\nouppercase{\leftmark}}
				}
			}
			\ifthenelse{\equal{\hfwidthwrap}{true}}{
				\fancyhead[L]{
					\begin{minipage}[t]{\hfwidthtitle\linewidth}
						\begin{flushleft}
							\nouppercase{\leftmark}
						\end{flushleft}
					\end{minipage}
				}
			}{
				\fancyhead[L]{\nouppercase{\leftmark}}
			}
		}
		
		% Actualiza los header-footer
		\ifthenelse{\equal{\hfstyle}{style1}}{
			\COREhfstyledefA
		}{
		\ifthenelse{\equal{\hfstyle}{style1-i}}{ % impar izquierdo
			\fancypagestyle{plain}{\fancyhead[LE,RO]{\nouppercase{\leftmark}}}
			\fancyhead[LE,RO]{\nouppercase{\leftmark}}
		}{
		\ifthenelse{\equal{\hfstyle}{style1-d}}{ % impar derecho
			\fancypagestyle{plain}{\fancyhead[LO,RE]{\nouppercase{\leftmark}}}
			\fancyhead[LO,RE]{\nouppercase{\leftmark}}
		}{
		\ifthenelse{\equal{\hfstyle}{style2}}{
			\COREhfstyledefA
		}{
		\ifthenelse{\equal{\hfstyle}{style2-i}}{ % impar izquierdo
			\fancypagestyle{plain}{\fancyhead[LE,RO]{\nouppercase{\leftmark}}}
			\fancyhead[LE,RO]{\nouppercase{\leftmark}}
		}{
		\ifthenelse{\equal{\hfstyle}{style2-d}}{ % impar derecho
			\fancypagestyle{plain}{\fancyhead[LO,RE]{\nouppercase{\leftmark}}}
			\fancyhead[LO,RE]{\nouppercase{\leftmark}}
		}{
		\ifthenelse{\equal{\hfstyle}{style4}}{
			\COREhfstyledefA
		}{
		\ifthenelse{\equal{\hfstyle}{style5}}{
			\COREhfstyledefB
		}{
		\ifthenelse{\equal{\hfstyle}{style5-d}}{ % impar derecho
			\COREhfstyledefB
		}{
		\ifthenelse{\equal{\hfstyle}{style5-i}}{ % impar izquierdo
			\COREhfstyledefB
		}{
		\ifthenelse{\equal{\hfstyle}{style9}}{
			\COREhfstyledefA
		}{
		\ifthenelse{\equal{\hfstyle}{style9-d}}{ % impar derecho
			\COREhfstyledefA
		}{
		\ifthenelse{\equal{\hfstyle}{style9-i}}{ % impar izquierdo
			\COREhfstyledefA
		}{
		\ifthenelse{\equal{\hfstyle}{style10}}{
			\COREhfstyledefC
		}{
		\ifthenelse{\equal{\hfstyle}{style10-d}}{ % impar derecho
			\COREhfstyledefC
		}{
		\ifthenelse{\equal{\hfstyle}{style10-i}}{ % impar izquierdo
			\COREhfstyledefC
		}{
		\ifthenelse{\equal{\hfstyle}{style11}}{ % Similar a 1
			\COREhfstyledefA
		}{
		\ifthenelse{\equal{\hfstyle}{style14}}{ % Similar a 4
			\COREhfstyledefA
		}{
		\ifthenelse{\equal{\hfstyle}{style15}}{ % Similar a 1
			\COREhfstyledefA
		}{
			% No se encontró el header-footer, no hace nada
		}}}}}}}}}}}}}}}}}}}
	}{
	}

	% -----------------------------------------------------------------------------
	% Estilo de títulos - reestablece estilos por el índice
	% -----------------------------------------------------------------------------
	\sectionfont{\color{\titlecolor} \fontsizetitle \styletitle \selectfont}
	\subsectionfont{\color{\subtitlecolor} \fontsizesubtitle \stylesubtitle \selectfont}
	\subsubsectionfont{\color{\subsubtitlecolor} \fontsizesubsubtitle \stylesubsubtitle \selectfont}
	\titleformat{\subsubsubsection}{\color{\ssstitlecolor} \normalfont \fontsizessstitle \stylessstitle}{\thesubsubsubsection}{1em}{}
	\titlespacing*{\subsubsubsection}{0pt}{3.25ex plus 1ex minus .2ex}{1.5ex plus .2ex}

	% -----------------------------------------------------------------------------
	% Crea funciones para numerar objetos
	% -----------------------------------------------------------------------------

	% Numeración de la sección en los objetos CÓDIGO FUENTE
	\ifthenelse{\equal{\showsectioncaptioncode}{none}}{
		\def\sectionobjectnumcode {}
	}{
	\ifthenelse{\equal{\showsectioncaptioncode}{sec}}{
		\def\sectionobjectnumcode {\thesection\sectioncaptiondelimiter}
	}{
	\ifthenelse{\equal{\showsectioncaptioncode}{ssec}}{
		\def\sectionobjectnumcode {\thesubsection\sectioncaptiondelimiter}
	}{
	\ifthenelse{\equal{\showsectioncaptioncode}{sssec}}{
		\def\sectionobjectnumcode {\thesubsubsection\sectioncaptiondelimiter}
	}{
	\ifthenelse{\equal{\showsectioncaptioncode}{ssssec}}{
		\ifthenelse{\equal{\showdotaftersnum}{true}}{
			\def\sectionobjectnumcode {\thesubsubsubsection}
		}{
			\def\sectionobjectnumcode {\thesubsubsubsection\sectioncaptiondelimiter}
		}
	}{
	\ifthenelse{\equal{\showsectioncaptioncode}{chap}}{
		\def\sectionobjectnumcode {\thechapter\sectioncaptiondelimiter}
	}{
		\throwbadconfig{Valor configuracion incorrecto}{\showsectioncaptioncode}{none,chap,sec,ssec,sssec,ssssec}}}}}}
	}

	% Numeración de la sección en los objetos ECUACIONES
	\ifthenelse{\equal{\showsectioncaptioneqn}{none}}{
		\def\sectionobjectnumeqn {}
	}{
	\ifthenelse{\equal{\showsectioncaptioneqn}{sec}}{
		\def\sectionobjectnumeqn {\thesection\sectioncaptiondelimiter}
	}{
	\ifthenelse{\equal{\showsectioncaptioneqn}{ssec}}{
		\def\sectionobjectnumeqn {\thesubsection\sectioncaptiondelimiter}
	}{
	\ifthenelse{\equal{\showsectioncaptioneqn}{sssec}}{
		\def\sectionobjectnumeqn {\thesubsubsection\sectioncaptiondelimiter}
	}{
	\ifthenelse{\equal{\showsectioncaptioneqn}{ssssec}}{
		\ifthenelse{\equal{\showdotaftersnum}{true}}{
			\def\sectionobjectnumeqn {\thesubsubsubsection}
		}{
			\def\sectionobjectnumeqn {\thesubsubsubsection\sectioncaptiondelimiter}
		}
	}{
	\ifthenelse{\equal{\showsectioncaptioneqn}{chap}}{
		\def\sectionobjectnumeqn {\thechapter\sectioncaptiondelimiter}
	}{
		\throwbadconfig{Valor configuracion incorrecto}{\showsectioncaptioneqn}{none,chap,sec,ssec,sssec,ssssec}}}}}}
	}

	% Numeración de la sección en los objetos FIGURAS
	\ifthenelse{\equal{\showsectioncaptionfig}{none}}{
		\def\sectionobjectnumfig {}
	}{
	\ifthenelse{\equal{\showsectioncaptionfig}{sec}}{
		\def\sectionobjectnumfig {\thesection\sectioncaptiondelimiter}
	}{
	\ifthenelse{\equal{\showsectioncaptionfig}{ssec}}{
		\def\sectionobjectnumfig {\thesubsection\sectioncaptiondelimiter}
	}{
	\ifthenelse{\equal{\showsectioncaptionfig}{sssec}}{
		\def\sectionobjectnumfig {\thesubsubsection\sectioncaptiondelimiter}
	}{
	\ifthenelse{\equal{\showsectioncaptionfig}{ssssec}}{
		\ifthenelse{\equal{\showdotaftersnum}{true}}{
			\def\sectionobjectnumfig {\thesubsubsubsection}
		}{
			\def\sectionobjectnumfig {\thesubsubsubsection\sectioncaptiondelimiter}
		}
	}{
	\ifthenelse{\equal{\showsectioncaptionfig}{chap}}{
		\def\sectionobjectnumfig {\thechapter\sectioncaptiondelimiter}
	}{
		\throwbadconfig{Valor configuracion incorrecto}{\showsectioncaptionfig}{none,chap,sec,ssec,sssec,ssssec}}}}}}
	}

	% Numeración de la sección en los objetos TABLAS
	\ifthenelse{\equal{\showsectioncaptiontab}{none}}{
		\def\sectionobjectnumtab {}
	}{
	\ifthenelse{\equal{\showsectioncaptiontab}{sec}}{
		\def\sectionobjectnumtab {\thesection\sectioncaptiondelimiter}
	}{
	\ifthenelse{\equal{\showsectioncaptiontab}{ssec}}{
		\def\sectionobjectnumtab {\thesubsection\sectioncaptiondelimiter}
	}{
	\ifthenelse{\equal{\showsectioncaptiontab}{sssec}}{
		\def\sectionobjectnumtab {\thesubsubsection\sectioncaptiondelimiter}
	}{
	\ifthenelse{\equal{\showsectioncaptiontab}{ssssec}}{
		\ifthenelse{\equal{\showdotaftersnum}{true}}{
			\def\sectionobjectnumtab {\thesubsubsubsection}
		}{
			\def\sectionobjectnumtab {\thesubsubsubsection\sectioncaptiondelimiter}
		}
	}{
	\ifthenelse{\equal{\showsectioncaptiontab}{chap}}{
		\def\sectionobjectnumtab {\thechapter\sectioncaptiondelimiter}
	}{
		\throwbadconfig{Valor configuracion incorrecto}{\showsectioncaptiontab}{none,chap,sec,ssec,sssec,ssssec}}}}}}
	}

	% -----------------------------------------------------------------------------
	% Modifica numeración de objetos
	% -----------------------------------------------------------------------------

	% Código fuente, INCLUIR SECCIÓN
	\ifthenelse{\equal{\captionnumcode}{arabic}}{
		\renewcommand{\thelstlisting}{\sectionobjectnumcode\arabic{lstlisting}}
	}{
	\ifthenelse{\equal{\captionnumcode}{alph}}{
		\renewcommand{\thelstlisting}{\sectionobjectnumcode\alph{lstlisting}}
	}{
	\ifthenelse{\equal{\captionnumcode}{Alph}}{
		\renewcommand{\thelstlisting}{\sectionobjectnumcode\Alph{lstlisting}}
	}{
	\ifthenelse{\equal{\captionnumcode}{roman}}{
		\renewcommand{\thelstlisting}{\sectionobjectnumcode\roman{lstlisting}}
	}{
	\ifthenelse{\equal{\captionnumcode}{Roman}}{
		\renewcommand{\thelstlisting}{\sectionobjectnumcode\Roman{lstlisting}}
	}{
		\throwbadconfig{Tipo numero codigo fuente desconocido}{\captionnumcode}{arabic,alph,Alph,roman,Roman}}}}}
	}

	% Ecuaciones, INCLUIR SECCIÓN
	\ifthenelse{\equal{\captionnumequation}{arabic}}{
		\renewcommand{\theequation}{\sectionobjectnumeqn\arabic{equation}}
	}{
	\ifthenelse{\equal{\captionnumequation}{alph}}{
		\renewcommand{\theequation}{\sectionobjectnumeqn\alph{equation}}
	}{
	\ifthenelse{\equal{\captionnumequation}{Alph}}{
		\renewcommand{\theequation}{\sectionobjectnumeqn\Alph{equation}}
	}{
	\ifthenelse{\equal{\captionnumequation}{roman}}{
		\renewcommand{\theequation}{\sectionobjectnumeqn\roman{equation}}
	}{
	\ifthenelse{\equal{\captionnumequation}{Roman}}{
		\renewcommand{\theequation}{\sectionobjectnumeqn\Roman{equation}}
	}{
		\throwbadconfig{Tipo numero ecuacion desconocido}{\captionnumequation}{arabic,alph,Alph,roman,Roman}}}}}
	}

	% Figuras, INCLUIR SECCIÓN
	\ifthenelse{\equal{\captionnumfigure}{arabic}}{
		\renewcommand{\thefigure}{\sectionobjectnumfig\arabic{figure}}
	}{
	\ifthenelse{\equal{\captionnumfigure}{alph}}{
		\renewcommand{\thefigure}{\sectionobjectnumfig\alph{figure}}
	}{
	\ifthenelse{\equal{\captionnumfigure}{Alph}}{
		\renewcommand{\thefigure}{\sectionobjectnumfig\Alph{figure}}
	}{
	\ifthenelse{\equal{\captionnumfigure}{roman}}{
		\renewcommand{\thefigure}{\sectionobjectnumfig\roman{figure}}
	}{
	\ifthenelse{\equal{\captionnumfigure}{Roman}}{
		\renewcommand{\thefigure}{\sectionobjectnumfig\Roman{figure}}
	}{
		\throwbadconfig{Tipo numero figura desconocido}{\captionnumfigure}{arabic,alph,Alph,roman,Roman}}}}}
	}

	% Subfiguras, NO USAR SECCIONES YA QUE SON HIJAS DE FIGURA
	\ifthenelse{\equal{\captionnumsubfigure}{arabic}}{
		\renewcommand{\thesubfigure}{\arabic{subfigure}}
	}{
	\ifthenelse{\equal{\captionnumsubfigure}{alph}}{
		\renewcommand{\thesubfigure}{\alph{subfigure}}
	}{
	\ifthenelse{\equal{\captionnumsubfigure}{Alph}}{
		\renewcommand{\thesubfigure}{\Alph{subfigure}}
	}{
	\ifthenelse{\equal{\captionnumsubfigure}{roman}}{
		\renewcommand{\thesubfigure}{\roman{subfigure}}
	}{
	\ifthenelse{\equal{\captionnumsubfigure}{Roman}}{
		\renewcommand{\thesubfigure}{\Roman{subfigure}}
	}{
		\throwbadconfig{Tipo numero subfigura desconocido}{\captionnumsubfigure}{arabic,alph,Alph,roman,Roman}}}}}
	}

	% Tablas, INCLUIR SECCIÓN
	\ifthenelse{\equal{\captionnumtable}{arabic}}{
		\renewcommand{\thetable}{\sectionobjectnumtab\arabic{table}}
	}{
	\ifthenelse{\equal{\captionnumtable}{alph}}{
		\renewcommand{\thetable}{\sectionobjectnumtab\alph{table}}
	}{
	\ifthenelse{\equal{\captionnumtable}{Alph}}{
		\renewcommand{\thetable}{\sectionobjectnumtab\Alph{table}}
	}{
	\ifthenelse{\equal{\captionnumtable}{roman}}{
		\renewcommand{\thetable}{\sectionobjectnumtab\roman{table}}
	}{
	\ifthenelse{\equal{\captionnumtable}{Roman}}{
		\renewcommand{\thetable}{\sectionobjectnumtab\Roman{table}}
	}{
		\throwbadconfig{Tipo numero tabla desconocido}{\captionnumtable}{arabic,alph,Alph,roman,Roman}}}}}
	}

	% Subtablas, NO INCLUIR SECCIÓN YA QUE SON HIJAS DE LAS TABLAS
	\ifthenelse{\equal{\captionnumsubtable}{arabic}}{
		\renewcommand{\thesubtable}{\arabic{subtable}}
	}{
	\ifthenelse{\equal{\captionnumsubtable}{alph}}{
		\renewcommand{\thesubtable}{\alph{subtable}}
	}{
	\ifthenelse{\equal{\captionnumsubtable}{Alph}}{
		\renewcommand{\thesubtable}{\Alph{subtable}}
	}{
	\ifthenelse{\equal{\captionnumsubtable}{roman}}{
		\renewcommand{\thesubtable}{\roman{subtable}}
	}{
	\ifthenelse{\equal{\captionnumsubtable}{Roman}}{
		\renewcommand{\thesubtable}{\Roman{subtable}}
	}{
		\throwbadconfig{Tipo numero subtabla desconocido}{\captionnumsubtable}{arabic,alph,Alph,roman,Roman}}}}}
	}

	% -----------------------------------------------------------------------------
	% Se reestablecen números de página y secciones
	% -----------------------------------------------------------------------------

	% Se usa número de páginas en arábigo si es que se tenía activado los númeos romanos
	\ifthenelse{\equal{\predocpageromannumber}{true}}{
		\renewcommand{\thepage}{\arabic{page}}}{
	}

	% Reinicia número de página
	\ifthenelse{\equal{\predocresetpagenumber}{true}}{
		\setcounter{page}{1}}{
	}
	\setcounter{section}{0}
	\setcounter{footnote}{0}

	% -----------------------------------------------------------------------------
	% Muestra los números de línea
	% -----------------------------------------------------------------------------
	\ifthenelse{\equal{\showlinenumbers}{true}}{
		\linenumbers}{
	}

	% -----------------------------------------------------------------------------
	% Establece el estilo de las subsubsubsecciones
	% -----------------------------------------------------------------------------
	\titleclass{\subsubsubsection}{straight}[\subsection]

}

% END