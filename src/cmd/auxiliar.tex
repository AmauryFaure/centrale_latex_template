% Template:     Template Auxiliares LaTeX
% Documento:    Funciones exclusivas de Template-Auxiliar
% Versión:      DEV
% Codificación: UTF-8
%
% Autor: Pablo Pizarro R. @ppizarror
%        Facultad de Ciencias Físicas y Matemáticas
%        Universidad de Chile
%        pablo@ppizarror.com
%
% Sitio web del proyecto: [https://ppizarror.com/Template-Auxiliares/]
% Licencia: MIT           [https://opensource.org/licenses/MIT]

% COMPILACION
% - environments.tex -> {references, anexo, columna}

% Insertar nuevo título de pregunta
%	#1	Título
\newcommand{\newquestion}[1]{
	\emptyvarerr{\newquestion}{#1}{Titulo pregunta no definido}
	\sectionanum{#1}
}

% Insertar nuevo título de pregunta encerrado en un recuadro
%	#1	Número pregunta
\newcommand{\newboxquestion}[1]{
	\emptyvarerr{\newquestion}{#1}{Titulo pregunta no definido}
	\phantomsection
	\newp \fbox{\ \textbf{#1}.-\ } \noindent
	\pdfbookmark[1]{#1}{toc}
}

% Crea una sección de imágenes múltiples
%	#1	Label (opcional)
%	#2	Caption
\newenvironment{images}[2][]{
	% Modifica globales
	\def\envimageslabelvar {#1}
	\def\envimagescaptionvar {#2}
	\def\GLOBALenvimageinitialized {true}
	\def\GLOBALenvimageadded {false}
	
	% Configura caption y márgenes
	\vspace{\marginimagetop cm}
	\setcaptionmargincm{\captionmarginmultimg}
	
	% Inicia la figura
	\begin{figure}[H] \centering%
	\thisfloatpagestyle{fancy}%
		\vspace{\marginimagemulttop cm}%
		}{
		\setcaptionmargincm{\captionlrmargin}%
		\ifthenelse{\equal{\envimagescaptionvar}{}}{ % \ifx\hfuzz no sirve
			\vspace{\captionlessmarginimage cm}%
		}{%
			\caption{\envimagescaptionvar\envimageslabelvar}%
		}%
	\end{figure}%

	% Reestablece caption y márgenes
	\setcaptionmargincm{\captionlrmargin}
	\vspace{\marginimagebottom cm}
	
	% Reestablece globales
	\def\GLOBALenvimageinitialized {false}
}

% END